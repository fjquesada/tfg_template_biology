% !TeX spellcheck = es_ES
\documentclass[12pt, a4paper, twoside]{report}   % Tamaño de papel, de fuente y márgenes
\usepackage[spanish]{babel}  % Idioma y codificación, para poder usar
\usepackage[utf8x]{inputenc} % tildes, eñes, etc., sin problemas
\usepackage[T1]{fontenc}
\usepackage{pdfpages}
% ============================================================|
%                                                             |
% Plantilla para TFG del Dpto. de Informática de la EPSJ v1.0 |
%                                                             |
% Ajustada según la normativa de estilo indicada por la EPSJ  |
% en el documento https://eps.ujaen.es/sites/centro_epsj/files|
% /uploads/documents/grados/TFG/criteriosYestilo_TFG.pdf      |
%                                                             |
%    - A4 con márgenes de 2.5cm                               |
%    - Intelineado de 1.5 líneas                              |
%    - Primera línea de cada párrafo con sangrado             |
%    - Fuente Arial de 12pt                                   |
%    - Cabecera con nombre de estudiante y título de TFG      |
%    - Pie con nombre del centro y nº de página               |
%    - Portada oficial EPSJ - Dpto. Informática con datos     |
%                                                             |
% 2020/09/25 - Francisco Charte Ojeda                         |
% CC0 1.0 Universal (CC0 1.0)                                 |
% https://creativecommons.org/publicdomain/zero/1.0/deed.es   |
%                                                             |
%=============================================================|
\usepackage{sectsty}
\subsectionfont{\normalfont\itshape}

% ==== Introducir aquí el nombre del estudiante
\def\Estudiante{Nombre Apellido Apellido}

% ==== Introducir aquí el nombre de los tutores. Si solo hay uno dejar las llaves de \TutorB vacías
\def\TutorA{Nombre Apellido Apellido}
\def\TutorB{}

% ==== Introducir aquí el título de completo y abreviado (para las cabeceras) del TFG
\def\TituloTFG{Título}
\def\TituloAbreviado{Título abreviado}

% ==== Introducir aquí el mes y año de presentación del TFG
\def\Fecha{Día de Mes de Año}

\input{Portada.tex}
\renewcommand{\arraystretch}{1.4}
\renewcommand{\baselinestretch}{1.5}
\begin{document}
    
    \Portada~			

    \pagenumbering{roman}  % Numeración romana para los agradecimientos, dedicatoria y tablas de contenidos

    %\input{Agradecimientos.tex} % Editar este archivo para introducir los agradecimientos/dedicatoria

    \tableofcontents  % Tabla de contenidos

% ===== Comentar o eliminar las líneas de los índices que no deseen incluirse al inicio de la memoria

    \clearpage\thispagestyle{empty}\cleardoublepage
    \listoffigures		% Índice de figuras

    \clearpage\thispagestyle{empty}\cleardoublepage
    \listoftables 		% Índice de tablas

    \clearpage\thispagestyle{empty}\cleardoublepage
    \pagenumbering{arabic} % Numeración arábiga para el resto del documento

% ===== Archivos LaTeX con los distintos capítulos que componen la memoria

\chapter*{RESUMEN}

...


\section*{PALABRAS CLAVE}
Palabra clave 1, Palabra clave 2 ...
\chapter*{ABSTRACT}

...


\section*{KEYWORDS}
Keyword1, Keyword2

\input{chapters/antecedentes}
\chapter{METODOLOGÍA DE INVESTIGACIÓN}

Metodología de Investigación...


\chapter{RESULTADO DE LA INVESTIGACIÓN}

Resultado ...


\chapter{DISCUSIÓN DE INVESTIGACIÓN}

Discusión ...
\input{chapters/conclusiones}
\clearpage\thispagestyle{empty}\cleardoublepage

% ===== Configuración de la bibliografía

\pagenumbering{roman}  % Numeración romana para el índice
\bibliographystyle{plainnat}
\bibliography{bibliografia}
\addcontentsline{toc}{chapter}{Bibliografía}

\end{document}